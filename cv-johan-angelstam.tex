\documentclass{cv-stylish}
\usepackage{cv-stylish}
\usepackage{changepage}
\usepackage{fontawesome5}

\hyphenation{develop-ment}

%---------------------------------------------------------------------
%  FOOTER SECTION
%---------------------------------------------------------------------

\footer{
Gammelbo gård 338 {\large\textperiodcentered} 711 98 Ramsberg {\large\textperiodcentered} Sweden\\ % Your mailing address
{\Large\Letter} johan@angelstam.se \ {\Large\Telefon} +46(0)70-308 24
94 \\ % Your email address and phone number
}

%---------------------------------------------------------------------

\begin{document}

\begin{center} % Center everything in the document

%---------------------------------------------------------------------
%  HEADER SECTION
%---------------------------------------------------------------------

\header{Johan Angelstam} % Your name at the top

\vspace{1.0cm} % Extra whitespace after the large name at the top

%---------------------------------------------------------------------
%	OBJECTIVE
%---------------------------------------------------------------------

%\section{Objective}

\begin{adjustwidth}{1cm}{1cm}
%A position in the field of computers with special interests in
%business applications \\ programming, information processing, and
%management systems.

% Först: erfarenhet, utbildning för att ge den) som läser en mycket
% generell bild av vem du är.
% Sedan: konkretisera och specificera dina viktigaste erfarenheter och
% kompetenser.
%
% From Sofie Hill, TRR Göteborg to Everyone:
% 1. Profession (dina relevanta kunskaper och erfarenheter).
% 2. Person (vem du är som person på jobbet/får energi av).
% 3. Ambition (vad du vill göra/jobba med. Gärna få med titlar du vill bli hittad på).
% 4. Nyckelord (tänk dina främsta kompetenser).
% 5. Kontaktuppgifter (mail och gärna telefonnummer).


I am a software engineer with more than 20 years of experience in
both front-end and back-end development, with a strong background
working on both new and legacy systems.
I have also been developing desktop applications and have some
experience working with embedded systems.
I pay great attention to
detail and I am both persistent and determined when working towards
clear and well-motivated goals.
I like finding easy to understand solutions to user need and I'm good
at finding alternate solutions that has not been considered.
 I have been involved in projects of
all sizes, from small companies where being pragmatic and flexible is
key, to process-heavy projects towards telecom. I am skilled in developing
microservices and automated testing. I am always
eager to learn and adapt to new technologies and methodologies and I am
also passionate about sharing my knowledge and experience with my team
and community. My big interest right now are around software development
methodologies, usability, DevOps, CI/CD and automation.

% Experienced integrating many different kinds of systems and API:s.

% I am easy going and can adapt to the situation. I pride myself in
% being a fast learner and finding the essence of a subject. This along
% with my drive for for improvements has made me into an adapt software
% engineer who is not afraid to venture out of my comfort zone trying
% new things.

% I'm a full-stack developer with extensive experience in both front-end
% and back-end development.

\end{adjustwidth}

%---------------------------------------------------------------------
%	WORK EXPERIENCE
%---------------------------------------------------------------------

\section{Work Experience}


\begin{JobTable}
  % Org.nr: 556179-5161
  June 2021 --- September 2023 & \hfill Full-Remote, Stockholm, Sweden \\
  \multicolumn{2}{X}{\hspace{5mm} \textbf{Giesecke+Devrient Mobile Security Sweden AB}} \\[3pt]
  \hspace{5mm} \textbf{Software Engineer}
  & \hfill Java · Spring Boot · Oracle · Jenkins · Ansible · Docker · Katalon \\
\end{JobTable}
\begin{InfoBody}
% I have gathered work experience in the following areas:
Giesecke+Devrient (G+D) is a global company that specializes in
security technology, particularly in the areas of payment,
connectivity, identities, and digital infrastructures.

As part of a team that implements and test new features in G+D Mobile
Security's GSMA compliant back-end for managing eSIM profiles for
remote provisioning, I have:

\begin{compactitem}
  \item Developed new features using Java with Spring Boot and Oracle in
    microservices.
  \item Developed automated tests for new features in Unit, Integration and
    System testing levels using Java, Spring Boot, Katalon and Docker.
  \item Improved and maintained our continuous integration, delivery
    and automated testing pipelines using Jenkins multi-branch
    pipeline and Bitbucket.
  \item Analyzed and improved code safety and security after routine
    checks using SonarQube.
  \item Migrated the Java 8 code base to Java 17.
  \item Logging and monitoring with Splunk and Grafana for observability.
  \item Performance test deployment in the Cleura cloud platform.
  \item Cooperated remotely with my Scrum team and with collegues in
    Sweden and Germany.
\end{compactitem}
Good references available upon request.
\end{InfoBody}

\vspace{1em}
\begin{JobTable}
  October 2015 --- September 2020 & \hfill Lindesberg, Sweden \\[3pt]
  \multicolumn{2}{X}{\hspace{5mm} \textbf{Followit Sweden AB}} \\[3pt]
  \hspace{5mm} \textbf{Software Engineer}
  & \hfill C\# · TypeScript · PHP · MySQL · MSSQL · JavaScript · C \\
  % \multicolumn{2}{X}{\hspace{5mm}\em{Systemutvecklare och programvarukonstruktör}} \\
\end{JobTable}
\begin{InfoBody}
% I have gathered work experience in the following areas:
% Utveckling av webbapplikationer byggda med Angular och
% Typescript. API:er byggda med JSON-baserat REST API mot existerande
% back-end. Back-end byggda med C\#/SQL Server och PHP/MySQL och
% integrationer där i mellan. Drift av Linux- och
% Windows-servrar. C\#, Javascript, Node.js, PHP, HTML/CSS, Bash,
% Git, Typescript, Angular, MySQL, SQL, APIs, Linux, C\#, Nancy, NancyFx,
% PHP

After my university studies in Linköping ended in 2015 the organization
had changed some and I got new responsibilites.

\begin{compactitem}
  \item Architecting, developing and supporting: customer interfaces in
    C\# and TypeScript; back-ends in C\# and PHP; and databases for
    GPS-tracking units.
  \item Developing databases and interfaces for managing end-customer
    subscriptions and b2b subscriptions packaged to end-customers.
  \item Integrations with for example Google Maps, a system for managing
    SIM-subscriptions and financial systems Visma and Fortnox.
  \item Setting up a CI/CD pipeline using Gitolite and Jenkins for the C\#
    back-end of a mobile app.
  \item Developing user interface and back-end for production support
    tools used for tracking quality and indentity of PCBs and GPS-tracking
    units in C\# WinForms, PHP and Filemaker.
  \item Managing the company's Git server and educating others on the
    internal use of version control and Git.
  \item Optimized MySQL storage performance using ZFS.
  \item Deployments on VMWare environments and on dedicated Linux and
    Windows servers.
  \item Manage and configure VMware ESXi.
\end{compactitem}

In 2015 we started developing Tellus 3 a new product for wildlife
research. I developed Wildlife Geo, a new user interface more tailored
for wildlife research customers than the exising Geo.
Wildlife Geo was the customer interface for the new tracking product
Tellus 3 that was developed in parallell by my collegues.
The front-end of Wildlife Geo was created using Angular 2+ with
TypeScript and communicated with the existing databases and back-ends
for Geo and Tellus using a new REST API. I developed a bridge service
using the NancyFX framework in C\# that presented the REST API for the
front-end.

During the following years I got tasked with maintenance of Geo, a
generic system for asset tracking developed by the company.
Geo was developed using ASP.NET and C\# with a MS SQL database.
I worked with trouble shooting of the system and bug fixes in the code
and database.
Geo used ASP.NET Web Forms for user interface and communicated with
its back-end using XML SOAP to internal Web APIs. The system had
integrations with Google Maps, GSM positioning, SMS gateways and
more.

In 2017 we started developing a new product for tracking hunting dogs
and asset tracking, Contact GPS 3.
I developed a new app for Contact GPS 3 and made adjustment in the
existing back-end for the new hardware. The new app was
created using Angular 2+ with TypeScript and communicated with the
exising back-end and database used by previous versions of the app.
The app used Google Maps for showing the current and historical
positions. It also incorporated map data of Sweden from \emph{The Land
  Survey (Lantmäteriet)} and of Norway from \emph{Norwegian Mapping
  Authority (Statens kartverk)}.

In 2019 I started to work on an internal system for managing cellular and
satellite subscriptions bundled in the companys products. The system
was made to replace a previous register made using Filemaker. The
back-end used Sails.js, Node.js and a MySQL database and the front-end
used Angular 2+ with TypeScript.
The system integrated with subscription management APIs of mobile and
satellite operators, and other company systems for our products and
billing systems.

In 2019 I also created a Windows application for the production of
Contact GPS 3 using C\# and WinForms that stored data in a MySQL
database.
The application was used by the circuit board manufacturer to register
results of tests and measurements in production.

\end{InfoBody}

\vspace{1em}

\begin{JobTable}
  August 2006 --- September 2015 & \hfill Lindesberg, Sweden \\[3pt]
  \multicolumn{2}{X}{\hspace{5mm} \textbf{Followit Lindesberg AB}} \\[3pt]
  \hspace{5mm} \textbf{Software Developer}
  & \hfill PHP · MySQL · JavaScript · Java (J2ME) · C \\
\end{JobTable}
\begin{InfoBody}
% I have gathered work experience in the following areas:
% Utveckling av API, back-end och administrationsverktyg byggda med PHP/MySQL, för datainsamling från GSM- och Iridium-baserade sändare. Utveckling av app för spårning byggd med J2ME. Administration av Gitolite. Drift av Linux- och Windows-servrar och administration av Windows AD.<br/>Java/Scala, Javascript, PHP, HTML/CSS, Git, MySQL, APIs, Linux

When I got hired in 2006 I continued my work on the system for the
Tellus wildlife tracking product, that I had started during my
internship period.

\begin{compactitem}
  \item Creating three generations of systems for receiving data from
    GPS-tracking units sending data by SMS, GPRS and the Iridium
    commercial communication satellite system format SBD.
    The systems, written in PHP, store the received data in a MySQL
    database, contain a web based administration, customer website
    with maps using \emph{Google Maps}, \emph{WMS} and \emph{ArcIMS} map data, the ability to forward
    data to the customers e-mail and production support
    systems. Responsible for continuously supporting, running,
    maintaining and improving the systems, on Linux and Windows
    servers.
  \item I worked on marketing material and customer manuals in Adobe
    Illustrator, InDesign and Photoshop.
  \item Managing the company's Git server and educating others on the
    internal use of version control and Git.
  \item Some experience coding for Microchip PIC18 in C.
\end{compactitem}

Tellus GPS System is an animal tracking product for wildlife research
that communicates using GSM and Iridium. I created the system that
received, processed and stored the data sent by the trackers. The
system was created using PHP with HTML/CSS front-end and has
a MySQL database for storing customer information and data from the
trackers.

In 2007 we started to work on Contact GPS ``1/2'', a product for tracking
hunting dogs. I developed a back-end system in PHP with a MySQL
database and a user interface for customers and
administration/production.
I also developed a mobile app in J2ME that was used for displaying the
dogs current position on a map during a hunt to the customers.

\end{InfoBody}


%\vspace{12pt} % Empty line
% \begin{InfoTable}
%  Period & \textbf{May 2011 --- 2013 (Part Time)}\\
%  Employer & \textbf{Linköping University} \hfill Linköping, Sweden\\
%  Job Title & \textbf{Laboration Assistent}\\
% \end{InfoTable}
% \begin{tabularx}{0.97\linewidth}{X}
% Responsable for arranging study visits to different organizations and
% companies as a part of a course at the Department of Electrical
% Engineering.
% \end{tabularx}

%---------------------------------------------------------------------
%	EDUCATION
%---------------------------------------------------------------------

% \pagebreak

\section{Education}

\begin{InfoTable}
 Period & \textbf{August 2009 --- 2015}\\
 Program & \textbf{Master of Science in Computer
  Engineering} \em{Civilingenjör i datateknik}\\
 University & \textbf{Linköping University} \hfill Linköping, Sweden\\
%& Extra information about degree
\end{InfoTable}
\begin{InfoBody}
  The program teaches computer engineering related subjects like:
  \begin{compactitem}
    \item Software engineering and programming including:
      functional programming in Lisp, imperative programming in Ada,
      object oriented programming in Java and C++, data structures and
      algorithms, computer and software security.
    \item Electronics and computer hardware where you work on:
      designing and building amplifiers, digital systems design and
      building of logic systems, signal processing and control engineering.
    \item Mathematics including: calculus, linear algebra, discrete
      mathematics, logic, optimization, transform theory and statistics.
    \item Accounting and industrial economics.
  \end{compactitem}

  I specialized in international software engineering, in programming projects,
  software development methods and methodologies, databases, software
  security, usability, economics and project management.
% % I am currently working on finishing my master's thesis about user
% % interfaces for ontology alignment.
\end{InfoBody}

\vspace{10pt}

\begin{InfoTable}
 Period & \textbf{February 2014 --- July 2014}\\
 Program & \textbf{Exchange Semester: School of Software, International
   Software Engineering}\\
 University & \textbf{Harbin Institute of Technology} \hfill Harbin, Heilongjiang, China\\
%& Extra information about degree
\end{InfoTable}
\begin{InfoBody}
The exchange semester with Harbin Institute of Technology was a part
of joint master program between Linköping University and HiT. In
Harbin I studied with a group of students from Sweden, France, Ireland
and China. The curriculum was taught in english by professors from
different countries.
\end{InfoBody}

\vspace{10pt}

\begin{InfoTable}
 Period & \textbf{September 2013 --- December 2013}\\
 Internship & \textbf{Zenterio AB} \hfill Linköping, Sweden\\
%& Extra information about degree
\end{InfoTable}
\begin{InfoBody}
I worked with the \emph{Method and Tools} team, mainly on improvements
to the build system, \emph{Jenkins}, and on integration between Jenkins
and the company's SCM. I also made a case study about the company's
use of development support tools.
\end{InfoBody}

\vspace{10pt}

\begin{InfoTable}
 Period & \textbf{September 2003 --- 2006}\\
 Internship & \textbf{TVP Positioning AB} \hfill Lindesberg, Sweden\\
\end{InfoTable}
\begin{InfoBody}
Along with my upper secondary education I had an internship at a local
company.

I worked on development of the customer web interface for InCase using
HTML/CSS and PHP. InCase was a security product for finding missing
machines and vehicles. The web interface for the InCase system ran on
a Linux server and integrated with a SMS-gateway and an Web-API for
GSM positioning.

As my exam project I pioneered a customer web interface for a novel
system called Satlink, a GPS tracking collar for wildlife research on
larger animals, like elephants and bears. This interface was developed
with HTML/CSS and PHP and ran on a Windows server.

\end{InfoBody}

% \vspace{10pt}

% \begin{InfoTable}
%  Period & \textbf{August 2001 --- December 2006}\\
%  Degree & \textbf{Upper Secondary School - Natural Science}\\
%  School & \textbf{Lindeskolan} \hfill Lindesberg, Sweden\\
% %& Extra information about degree
% \end{InfoTable}


%---------------------------------------------------------------------
%	COURSES
%---------------------------------------------------------------------

\section{Courses}

\begin{InfoTable}
 Period & \textbf{May 2024 --- October 2024}\\
 Course & \textbf{Frontend Developer using React}\\
 School & \textbf{Lexicon IT-Proffs AB} \hfill Remote, Stockholm, Sweden \\
\end{InfoTable}
\begin{InfoBody}

The course teaches React for SPA web applications with State
Management and Routing, finishing with a project.
The course also give practice using current versions of:
HTML and CSS: Semantic HTML, CSS4 flexbox \& grid, responsive design using
media queries and SASS.
Github projects: SCRUM, agile development.
JavaScript: ECMAScript, TypeScript, Unobtrusive JavaScript, DOM Manipulation,
integration of CRUD APIs.

\end{InfoBody}

%---------------------------------------------------------------------
%	SKILLS
%---------------------------------------------------------------------

\section{Computer Skills}

\begin{tabular}{ @{} >{\bfseries}l @{\hspace{6ex}} l }
Computer Languages & C\#, Java, PHP, TypeScript, C, Lisp, VHDL, Ada, Lua \\
Frameworks & Spring Boot, Angular 2+, NancyFX, .NET MVC, WinForms, Sails.js, \\
         & Node.js, Smarty, (Twitter) Bootstrap, React \\
%Protocols \& APIs & XML, JSON, SOAP, REST \\
Databases & MySQL, MSSQL, Oracle, Filemaker, H2 \\
Operating Systems & Unix (Linux, Solaris, NetBSD), Mac OS, Windows \\
Tools & Git, Atom, Visual Studio, Latex, Emacs, Eclipse, IntelliJ,
        Xcode, Trac, \\
      &  Jenkins, Bitbucket, Jira, Confluence, SonarQube, Ansible, Postman
\end{tabular}


%---------------------------------------------------------------------
%	OTHER SKILLS
%---------------------------------------------------------------------

\section{Other}

\begin{InfoBody}
Board member of the local film club in Lindesberg since 2016. I play
movies selected by the club at the local cinema, participate in
selecting movies, maintain the clubs website and make posters for the
coming program.

Board member in the family bussiness.

I live in the country side with my family. To disconnect from
computer work I do physical work arround the house and in nearby
forrests. I woodwork, from cutting down trees to make my own lumber to
create things in and arround the home.
\end{InfoBody}


%---------------------------------------------------------------------
%	LINKS
%---------------------------------------------------------------------

\section{Links}
\begin{InfoBody}

\IconUrl{\faLinkedin}{www.linkedin.com/in/johan-angelstam} \hfill
\IconUrl{\faGithub}{github.com/angelstam}

\end{InfoBody}

\end{center}

\end{document}