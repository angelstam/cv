\documentclass{cv-stylish}
\usepackage{cv-stylish}
\usepackage{changepage}
\usepackage[swedish]{babel}

%---------------------------------------------------------------------
%  FOOTER SECTION
%---------------------------------------------------------------------

\footer{
Gammelbo gård 338 {\large\textperiodcentered} 711 98 Ramsberg {\large\textperiodcentered} Sverige\\ % Din postadress
{\Large\Letter} johan@angelstam.se \ {\Large\Telefon} +46(0)70-308 24
94 \\ % Din e-postadress och telefonnummer
}

%---------------------------------------------------------------------

\begin{document}

\begin{center} % Centrera allt i dokumentet

%---------------------------------------------------------------------
%  HEADER SECTION
%---------------------------------------------------------------------

\header{Johan Angelstam} % Ditt namn högst upp

\vspace{1.0cm} % Extra utrymme efter det stora namnet högst upp

%---------------------------------------------------------------------
%	MÅL
%---------------------------------------------------------------------

%\section{Mål}

\begin{adjustwidth}{1cm}{1cm}
% En position inom datavetenskap med särskilt intresse för
%företagsapplikationer, programmering, informationsbehandling och
%ledningssystem.

% Först: erfarenhet, utbildning för att ge den) som läser en mycket
% generell bild av vem du är.
% Sedan: konkretisera och specificera dina viktigaste erfarenheter och
% kompetenser.

Jag är mjukvaruutvecklare med mer än 20 års erfarenhet av
front-end och back-end-utveckling, med en gedigen erfarenhet
från både nya och äldre system.
Jag har också jobbat med utveckling av skrivbordsapplikationer och har
viss erfarenhet av inbyggda system.
Jag uppmärksam på
detaljer och är uthållig och målmedveten när jag arbetar mot
tydliga och väl motiverade mål. Jag har varit involverad i projekt av
alla storlekar, från små företag där pragmatism och flexibilitet
är nödvändigt, till processintensiva projekt inom telekommunikation.
Jag är skicklig på att utveckla mikrotjänster och automatiserad
testning. Jag gillar att lära mig om och använda nya teknologier
och metoder. Jag uppskattar att samarbeta med kollegor. Det är
viktigt för mig att jobba på en prestigelös arbetsplats där vi delar
våra kunskaper och erfarenheter med varandra.
% Just nu är mitt stora 
%intresse inom områdena DevOps, CI/CD och automatisering.

% Jag gillar att hitta enkla och förståeliga lösningar på användarbehov.

% Jag är anpassningsbar och kan anpassa mig till situationen. Jag är stolt
% över att vara en snabb inlärare och finna ämnets kärna. Tillsammans med
% mitt driv för förbättringar har jag utvecklats till en anpassningsbar
% mjukvaruingenjör som inte är rädd för att prova nya saker.

% Jag är en fullstack-utvecklare med omfattande erfarenhet inom både
% front-end och back-end-utveckling.

\end{adjustwidth}

%---------------------------------------------------------------------
%	ARBETSERFARENHET
%---------------------------------------------------------------------

\section{Arbetslivserfarenhet}

\begin{JobTable}
  % Org.nr: 556179-5161
  Juni 2021 --- September 2023 & \hfill Helt på distans, Stockholm, Sverige \\
  \multicolumn{2}{X}{\hspace{5mm} \textbf{Giesecke+Devrient Mobile Security Sweden AB}} \\[3pt]
  \hspace{5mm} \textbf{Mjukvaruutvecklare}
  & \hfill Java · Spring Boot · Oracle · Jenkins · Ansible · Docker · Katalon \\
\end{JobTable}
\begin{adjustwidth}{0.015\linewidth}{0.015\linewidth}
% Jag har samlat erfarenhet inom följande områden:
Giesecke+Devrient (G+D) är ett globalt företag som är specialiserat
inom säkerhetsteknik, särskilt inom områden som betalningar,
kommunikation, identiteter och digital infrastruktur.

Som en del av ett team som implementerar och testar nya funktioner i
G+D Mobile Securitys GSMA-kompatibla back-end för hantering av eSIM-profiler för fjärrprovisionering, har jag

\begin{compactitem}
  \item utvecklat nya funktioner med Java med Spring Boot och Oracle i mikrotjänster
  \item utvecklat automatiserade tester för nya funktioner på enhets-, integrations- och
    systemtestningsnivåer med Java, Spring Boot, Katalon och Docker
  \item förbättrat och underhållit våra pipelines för kontinuerlig integration, leverans och automatiserad testning med Jenkins multi-branch
    pipeline och Bitbucket
  \item analyserat och förbättrat kodens säkerhet och säkerhet efter rutinkontroller
\end{compactitem}
\end{adjustwidth}

\vspace{1em}
\begin{JobTable}
  Oktober 2015 --- September 2020 & \hfill Lindesberg, Sverige \\[3pt]
  \multicolumn{2}{X}{\hspace{5mm} \textbf{Followit Sweden AB}} \\[3pt]
  \hspace{5mm} \textbf{Systemutvecklare}
  & \hfill C\# · TypeScript · PHP · MySQL · MSSQL · JavaScript · C \\
  % \multicolumn{2}{X}{\hspace{5mm}\em{Systemutvecklare och programvarukonstruktör}} \\
\end{JobTable}
\begin{adjustwidth}{0.015\linewidth}{0.015\linewidth}
% Jag har samlat erfarenhet inom följande områden:
% Utveckling av webbapplikationer byggda med Angular och
% Typescript. API:er byggda med JSON-baserat REST API mot existerande
% back-end. Back-end byggda med C\#/SQL Server och PHP/MySQL och
% integrationer där i mellan. Drift av Linux- och
% Windows-servrar. C\#, Javascript, Node.js, PHP, HTML/CSS, Bash,
% Git, Typescript, Angular, MySQL, SQL, APIs, Linux, C\#, Nancy, NancyFx,
% PHP

\begin{compactitem}
  \item Arkitektur, utveckling och support av: kundgränssnitt i
    C\# och TypeScript; back-ends i C\# och PHP; och databaser för
    GPS-spårningsenheter.
  \item Utveckling av databaser och gränssnitt för hantering av
    slutkundsabonnemang och b2b-abonnemang som paketeras till slutkunder.
  \item Integrationer med exempelvis Google Maps, ett system för hantering
    av SIM-abonnemang och finansiella system som Visma och Fortnox.
  \item Etablering av en CI/CD-pipeline med Gitolite och Jenkins för C\#-back-enden av en mobilapp.
  \item Hantering av företagets Git-server och utbildning av andra om
    intern användning av versionskontroll och Git.
  \item Optimerade MySQL-lagringsprestanda med ZFS.
  \item Driftsatt på VMWare-miljöer och på dedikerade Linux- och
    Windows-servrar.
  \item Hantera och konfigurera VMware ESXi.
\end{compactitem}
\end{adjustwidth}

\vspace{1em}
\pagebreak
\begin{JobTable}
  Augusti 2006 --- September 2015 & \hfill Lindesberg, Sverige \\[3pt]
  \multicolumn{2}{X}{\hspace{5mm} \textbf{Followit Lindesberg AB}} \\[3pt]
  \hspace{5mm} \textbf{Mjukvaruutvecklare}
  & \hfill PHP · MySQL · JavaScript · Java (J2ME) · C \\
\end{JobTable}
\begin{adjustwidth}{0.015\linewidth}{0.015\linewidth}
% Jag har samlat erfarenhet inom följande områden:
% Utveckling av API, back-end och administrationsverktyg byggda med PHP/MySQL, för datainsamling från GSM- och Iridium-baserade sändare. Utveckling av app för spårning byggd med J2ME. Administration av Gitolite. Drift av Linux- och Windows-servrar och administration av Windows AD.<br/>Java/Scala, Javascript, PHP, HTML/CSS, Git, MySQL, APIs, Linux
\begin{compactitem}
  \item Skapat tre generationer av system för att ta emot data från
    GPS-spårningsenheter som skickar data via SMS, GPRS och Iridium
    kommersiell kommunikationssatellitsystem i formatet SBD.
    Systemen, skrivna i PHP, lagrar mottagen data i en MySQL-databas, innehåller en webbaserad administration, kundwebbplats
    med kartor med hjälp av \emph{Google Maps}, \emph{WMS} och \emph{ArcIMS}-kartdata, förmågan att vidarebefordra
    data till kundens e-post och produktionssupportsystem. Ansvarig för kontinuerligt stöd,
    drift, underhåll och förbättring av systemen, på Linux och Windows
    servrar.
  \item Hantera företagets Git-server och utbildning av andra om
    intern användning av versionskontroll och Git.
  %\item Some experience coding for Microchip PIC18 in C.
\end{compactitem}
\end{adjustwidth}

%\vspace{12pt} % Tom rad
% \begin{InfoTable}
%  Period & \textbf{Maj 2011 --- 2013 (Deltid)}\\
%  Arbetsgivare & \textbf{Linköpings universitet} \hfill Linköping, Sverige\\
%  Jobbtitel & \textbf{Laborationsassistent}\\
% \end{InfoTable}
% \begin{tabularx}{0.97\linewidth}{X}
% Ansvarig för att ordna studiebesök till olika organisationer och
% företag som en del av en kurs vid Institutionen för elektrisk
% ingenjörsvetenskap.
% \end{tabularx}

%---------------------------------------------------------------------
%	UTBILDNING
%---------------------------------------------------------------------

% \pagebreak

\section{Utbildning}

\begin{InfoTable}
 Period & \textbf{Augusti 2009 --- September 2015}\\
 Program & \textbf{Civilingenjör i datateknik}\\
 Universitet & \textbf{Linköpings universitet} \hfill Linköping, Sverige\\
%& Extra information om examen
\end{InfoTable}
% \begin{tabularx}{0.97\linewidth}{X}
% % Jag arbetar för närvarande med att avsluta mitt masterprojekt om användargränssnitt för ontologisk justering.
% \end{tabularx}

\vspace{10pt}

\begin{InfoTable}
 Period & \textbf{Februari 2014 --- Juli 2014}\\
 Program & \textbf{Utbytestermin: School of Software, International
   Software Engineering}\\
 Universitet & \textbf{Harbin Institute of Technology} \hfill Harbin, Heilongjiang, Kina\\
%& Extra information om examen
\end{InfoTable}

\vspace{10pt}

\begin{InfoTable}
 Period & \textbf{September 2013 --- December 2013}\\
 Praktik & \textbf{Zenterio AB} \hfill Linköping, Sverige\\
%& Extra information om examen
\end{InfoTable}
\begin{tabularx}{0.97\linewidth}{X}
Jag arbetade med \emph{Method and Tools}-teamet, främst med förbättringar
av byggsystemet, \emph{Jenkins}, och integrationen mellan Jenkins
och företagets SCM. Jag gjorde också en fallstudie om företagets
användning av utvecklingsstödsverktyg.
\end{tabularx}

% \vspace{10pt}

% \begin{InfoTable}
%  Period & \textbf{Augusti 2001 --- December 2006}\\
%  Grad & \textbf{Gymnasieskola - Naturvetenskap}\\
%  Skola & \textbf{Lindeskolan} \hfill Lindesberg, Sverige\\
% %& Extra information om examen
% \end{InfoTable}


%---------------------------------------------------------------------
%	COURSES
%---------------------------------------------------------------------

\section{Kurser}

\begin{InfoTable}
 Period & \textbf{May 2024 --- October 2024}\\
 Kurs & \textbf{Frontend-utvecklare med React}\\
 Utbildare & \textbf{Lexicon IT-Proffs AB} \hfill Distans, Stockholm, Sverige \\
\end{InfoTable}
\begin{InfoBody}

Kursen lär ut användning av React för att skapa SPA-webbapplikationer
med \emph{State Management} och \emph{Routing}, kursen innehåller
flera gruppuppgifter och avslutas med ett projekt.
Kursen ger också övning i att använda aktuella versioner av:
HTML och CSS: Semantisk HTML, \emph{CSS4 flexbox \& grid}, responsiv
webbdesign med hjälp av \emph{media queries} och SASS.
Github projekt: SCRUM och agil utveckling.
JavaScript: ECMAScript, TypeScript, \emph{Unobtrusive JavaScript},
DOM-manipulation och integration av CRUD API:er.

\end{InfoBody}

%---------------------------------------------------------------------
%	FÄRDIGHETER
%---------------------------------------------------------------------

\section{Datorfärdigheter}

\begin{tabular}{ @{} >{\bfseries}l @{\hspace{6ex}} l }
Datorspråk & C\#, Java, PHP, TypeScript, C, Lisp, VHDL, Ada \\
%Protokoll \& API:er & XML, JSON, SOAP, REST \\
Databaser & MySQL, MSSQL, Oracle, Filemaker, H2 \\
Operativsystem & Unix (Linux, Solaris, NetBSD), Mac OS, Windows \\
Verktyg & Git, Atom, Visual Studio, Latex, Emacs, Eclipse, IntelliJ,
        Xcode, Trac, \\
      &  Jenkins, Bitbucket, Jira, Confluence, SonarQube, Ansible, Postman
\end{tabular}

%---------------------------------------------------------------------

\end{center}

\end{document}
